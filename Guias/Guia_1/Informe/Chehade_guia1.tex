\documentclass[aps,prb,twocolumn,superscriptaddress,floatfix,longbibliography]{revtex4-2}

\usepackage[utf8]{inputenc}
\usepackage[spanish]{babel}
\usepackage{graphicx}
\usepackage{amsmath}
\usepackage{subcaption}
\usepackage{wrapfig} 
\usepackage[export]{adjustbox}

\usepackage{amsmath,amssymb} % math symbols
\usepackage{bm} % bold math font
\usepackage{graphicx} % for figures
\usepackage{comment} % allows block comments
\usepackage{textcomp} % This package is just to give the text quote '
%\usepackage{ulem} % allows strikeout text, e.g. \sout{text}

\usepackage[spanish]{babel}

\usepackage{enumitem}
\setlist{noitemsep,leftmargin=*,topsep=0pt,parsep=0pt}

\usepackage{xcolor} % \textcolor{red}{text} will be red for notes
\definecolor{lightgray}{gray}{0.6}
\definecolor{medgray}{gray}{0.4}

\usepackage{hyperref}
\hypersetup{
colorlinks=true,
urlcolor= blue,
citecolor=blue,
linkcolor= blue,
bookmarks=true,
bookmarksopen=false,
}

% Code to add paragraph numbers and titles
\newif\ifptitle
\newif\ifpnumber
\newcounter{para}
\newcommand\ptitle[1]{\par\refstepcounter{para}
{\ifpnumber{\noindent\textcolor{lightgray}{\textbf{\thepara}}\indent}\fi}
{\ifptitle{\textbf{[{#1}]}}\fi}}
%\ptitletrue  % comment this line to hide paragraph titles
%\pnumbertrue  % comment this line to hide paragraph numbers

% minimum font size for figures
\newcommand{\minfont}{6}

% Uncomment this line if you prefer your vectors to appear as bold letters.
% By default they will appear with arrows over them.
% \renewcommand{\vec}[1]{\bm{#1}}

%Cambiar Cuadros por Tablas y lista de...
%\renewcommand{\listtablename}{Índice de tablas}
\renewcommand{\tablename}{Tabla}
\renewcommand{\date}{Fecha}

\graphicspath{ {C:/Users/lupam/Mi unidad/Pablo Chehade/Instituto Balseiro (IB)/Métodos numéricos en fluidos I/Metodos_Num_Fluidos_I/Guias/Guia_1/Informe/Figures} } %Para importar imágenes desde una carpeta


\usepackage[bottom]{footmisc} %para que las notas al pie aparezcan en la misma página



\begin{comment}

%Comandos de interés:

* Para ordenar el documento:
\section{Introducción}
\section{\label{sec:Formatting}Formatting} %label para luego hacer referencia a esa sección

\ptitle{Start writing while you experiment} %pone nombre y título al documento dependiendo de si en el header están los comandos \ptitletrue y \pnumbertrue

* Ecuaciones:
\begin{equation}
a^2+b^2=c^2 \,.
\label{eqn:Pythagoras}
\end{equation}

* Conjunto de ecuaciones:
\begin{eqnarray}
\label{eqn:diagonal}
\nonumber d & = & \sqrt{a^2 + b^2 + c^2} \\
& = & \sqrt{3^2+4^2+12^2} = 13
\end{eqnarray}

* Para hacer items / enumerar:
\begin{enumerate}
  \item
\end{enumerate}

\begin{itemize}
  \item
\end{itemize}

* Figuras:
\begin{figure}[h]
    \includegraphics[clip=true,width=\columnwidth]{pixel-compare}
    \caption{}
     \label{fig:pixels}
\end{figure}

* Conjunto de figuras:
(no recuerdo)


* Para hacer referencias a fórmulas, tablas, secciones, ... dentro del documento:
\ref{tab:spacing}

* Para citar
Elementos de .bib
\cite{WhitesidesAdvMat2004}
url
\url{http://www.mendeley.com/}\\

* Agradecimientos:
\begin{acknowledgments}
We acknowledge advice from Jessie Zhang and Harry Pirie to produce Fig.\ \ref{fig:pixels}.
\end{acknowledgments}

* Apéndice:
\appendix
\section{\label{app:Mendeley}Mendeley}

* Bibliografía:
\bibliography{Hoffman-example-paper}

\end{comment}



\begin{document}

% Allows to rewrite the same title in the supplement
\newcommand{\mytitle}{Laboratorio 1}

\title{\mytitle}

\author{Pablo Chehade \\
    \small \textit{pablo.chehade@ib.edu.ar} \\
    \small \textit{Métodos Numéricos en Fluidos I, Instituto Balseiro, CNEA-UNCuyo, Bariloche, Argentina, 2022} \\}


\begin{abstract}

Se estudió el comportamiento de distintas aproximaciones numéricas en un problema de valores de contorno tipo Dirichlet de solución exacta conocida.

En primer lugar, se analizó cualitativamente ambas aproximaciones para $K$ y $N$ fijos. En segundo lugar, se estudió la dependencia del error local en un punto particular en función del espaciamiento $h$. Finalmente, se estudió cuántos puntos discretos $N$ son necesarios para obtener un determinado error global en función de las oscilaciones de la solución representadas por el parámetro $K$.

\end{abstract}

\maketitle

\textcolor{red}{
\begin{itemize}
    \item ¿Tengo que mencionar la computadora que usé? ¿En qué sección lo hago? NO. Nosotros no lo tenemos que poner. En los papers para cálculos grandes se suele poner.
    \item ¿Está bien el nombre de la sección "Método Numérico"?
    \item ¿Cuánto desarrollo hay que hacer en b y c?
    \item ¿Es necesario hacer el desarrollo para obtener la solución exacta o se puede dar por sabido? Puedo poner directamente la solución exacta.
\end{itemize}}

\section{Introducción}
\ptitle{En física, no todos los problemas tienen solución analítica, muchas veces es necesario recurrir a aproximaciones}

En ciencias físicas no todos los problemas tienen solución analítica \cite{Chule}. Debido a esto, es necesario recurrir a aproximaciones del mismo que sí posean solución o a esquemas numéricos que permitan aproximarlo computacionalmente. Sin embargo, estos esquemas no están excentos de error, por lo que deben ser estudiados en detalle para determinar su validez y aplicabilidad. Para esto es útil aplicar estos esquemas a problemas de solución exacta conocida.

\ptitle{En este trabajo se estudió el comportamiento de distintas aproximaciones numéricas en el siguiente problema de valores de contorno tipo Dirichlet}

En este trabajo se estudió el comportamiento de distintas aproximaciones numéricas en el siguiente problema de valores de contorno tipo Dirichlet
\begin{equation}
    \left\{\begin{matrix}
        \frac{d^2 y}{dx^2} - y = g(x), \\ 
        g (x) = - \sum_{j = 1}^K (1 + (j \pi)^2)\sin{(j \pi x)}, 0 \leq x \leq 1,\\
        y(0) = 0,\\
        y(1) = 1, \\
        \end{matrix}\right.
    \label{ec:PVC}
\end{equation}

con solución analítica
\begin{equation}
    y(x) = \frac{e^x - e^{-x}}{e - e^{-1}} + \sum_{j = 1}^K \sin{(j \pi x)}
    \label{ec:sol_analitica}
\end{equation}

\section{Método Numérico}

\ptitle{Discretización del dominio}

Para resolver numéricamente el problema de valores de contorno es necesario discretizar el dominio y proponer un esquema numérico que permita obtener la solución aproximada. El dominio se discretizó con puntos equiespaciados $x_i = i h$ donde $i = 1,\cdots, N$ y $h = 1/(N+1)$. En base a esto, el problema de valores iniciales \ref{ec:PVC} se puede escribir como
\begin{equation}
    \left\{\begin{matrix}
        y_i'' - y_i = g_i, i = 1, \cdots, N\\
        y_0 = y(0) = 0,\\
        y_{N+1} = y(1) = 1, \\
        \end{matrix}\right.
    \label{ec:PVC_numerico}
\end{equation}
donde $y_i''= \frac{d^2 y_i}{dx^2}$ y $g_i = g(x_i)$.

\ptitle{Para estimar y'' emplearon diferencias finitas centradas de segundo orden y el \textcolor{red}{esquema} de Padé \textcolor{red}{de 4to orden}}
Para estimar $y_i''$ se pueden utilizar distintos esquemas numéricos. En este trabajo se emplearon diferencias finitas centradas y la aproximación de Padé.


\subsection{Diferencias finitas centradas}
La fórmula de diferencias finitas centradas de segundo orden para la derivada segunda es \cite{Moin}
\begin{equation}
    y_i'' = \frac{y_{i+1} - 2 y_i + y_{i-1}}{h^2} + O(h^2).
    \label{ec:diferencias_centradas}
\end{equation}
Aplicándola a la ecuación diferencial \ref{ec:PVC_numerico} para los nodos internos $i = 2, \cdots, N-2$, se obtiene
\[ \frac{1}{h^2} y_{i-1} + \left (-\frac{2}{h^2} - 1 \right ) y_i + \frac{1}{h^2} y_{i-1} = g_i.\]
Para los nodos del borde se puede emplear la misma aproximación bajo la consideración de que $y_0 = y(0)$ e $y_{N+1} = y(1)$, es decir,
\[\frac{1}{h^2}y_2 + \left (-\frac{2}{h^2} - 1 \right ) y_1 = g_1 - \frac{1}{h^2}y(0)  \]
\[\frac{1}{h^2}y_N + \left (-\frac{2}{h^2} - 1 \right ) y_{N-1} = g_N - \frac{1}{h^2}y(1)  \]
El sistema de ecuaciones algebraicas anterior se puede escribir de la forma $A_{DC} \vec{y} = \vec{b}$, donde
\[
    A_{DC} = \left(\begin{matrix}
        -\frac{2}{h^2} - 1 & \frac{1}{h^2} & 0 & \cdots & 0 & 0\\
        \frac{1}{h^2} & -\frac{2}{h^2} - 1 & \frac{1}{h^2} & \cdots & 0 & 0\\
        0 & \frac{1}{h^2} & -\frac{2}{h^2} - 1 & \cdots & 0 & 0\\
        \vdots & \vdots & \vdots & \ddots & \vdots & \vdots\\
        0 & 0 & 0 & \cdots & -\frac{2}{h^2} - 1 & \frac{1}{h^2}\\
        0 & 0 & 0 & \cdots & \frac{1}{h^2} & -\frac{2}{h^2} - 1\\
        \end{matrix}\right),
\]
\[\vec{y} = (y_1, y_2, \cdots, y_N)\]
y
\[\vec{b} = (g_1 - y(0)/h^2, g_2, \cdots, g_{N-1}, g_N -  y(1)/h^2)\]


\subsection{Aproximación de Padé}

\ptitle{Obtención del sistema lineal en los nodos internos}
La fórmula de Padé de 4to orden para la derivada segunda es \cite{Moin}
\begin{equation}
    \frac{1}{12} y_{i-1}'' + \frac{10}{12} y_i'' + \frac{1}{12} y_{i+1}'' = \frac{y_{i+1} - 2 y_i + y_{i-1}}{h^2}
    \label{ec:Pade_interior}
\end{equation}
válida sólo para los nodos internos $i = 2, \cdots, N-2$.

\ptitle{Obtención del sistema lineal en el borde izquierdo}
Por otro lado, para los nodos del borde con $i = 1$ e $i = N$ es necesario derivar una aproximación de Padé. Para esto basta plantear
\begin{equation}
    y_1'' + b_2 y_2'' = a_0 y_0 + a_1 y_1 + a_2 y_2 +O(h^\alpha)
    \label{ec:Pade_izq_gral}
\end{equation}
y determinar los coeficientes $a_0$, $a_1$, $a_2$ y $b_2$ de modo de obtener el mayor orden de aproximación $\alpha$ posible. Para esto se desarrolla en Taylor $y_2''$, $y_0$ e $y_2$ alrededor de $x_1$. De este modo,
\begin{eqnarray}
    \nonumber y_2'' & = & y_1'' + h y_1''' + \frac{h^2}{2} y_1^{(IV)} + O(h^3), \\ 
    \nonumber y_2 & = & y_1 + h y_1' + \frac{h^2}{2} y_1'' + \frac{h^3}{6}y_1''' + \frac{h^4}{24} y_1^{(IV)} + O(h^3), \\
    \nonumber y_0 & = & y_1 - h y_1' + \frac{h^2}{2} y_1'' - \frac{h^3}{6}y_1''' + \frac{h^4}{24} y_1^{(IV)} + O(h^3),
\end{eqnarray}
donde se consideró que $a_2$ tiene unidades de $1/h^2$. Reemplazando estas expresiones en \ref{ec:Pade_izq} se obtiene
\begin{eqnarray}
    \nonumber y_1'' & = & y_1( a_0 + a_1 + a_2) + y_1'(-a_0 h + a_2 h) \\ 
    \nonumber & + & y_1''(a_0 \frac{h^2}{2} + a_2 \frac{h^2}{2} - b_2) + y_1'''(- a_0 \frac{h^3}{6} + a_2 \frac{h^3}{6} - b_2 h) \\
    \nonumber & + & y_1^{(IV)}(a_0 \frac{h^4}{24} + a_2 \frac{h^4}{24} - b_2 \frac{h^2}{2} ) + O(h^3).
\end{eqnarray}
Igualando ambos miembros da lugar al siguiente conjunto de ecuaciones para los coeficientes $a_0$, $a_1$, $a_2$ y $b_2$
\[\left\{\begin{matrix}
    a_0 + a_1 + a_2 = 0 \\
    h(-a_0 + a_2) = 0 \\
    a_0 \frac{h^2}{2} + a_2 \frac{h^2}{2} - b_2 = 1\\
    h(- a_0 \frac{h^2}{6} + a_2 \frac{h^2}{6} - b_2) = 0 \\
    \frac{h^2}{2}(a_0 \frac{h^2}{12} + a_2 \frac{h^2}{12} - b_2) = 0
    \end{matrix}\right.
\]
Solo es posible cumplir las cuatro primeras ecuaciones mediante la elección  $a_0 = 1/h^2$, $a_1 = -2/h^2$, $a_2 = 1/h^2$ y $b_2 = 0$. Al no ser posible anular el término de orden $h^2$, la aproximación de Padé resultante es de segundo orden. La expresión final es
\begin{equation}
    y_1'' = \frac{y_{2} - 2 y_1 + y_{0}}{h^2} + O(h^2).
    \label{eq:Pade_izq}
\end{equation}
que coincide con diferencias centradas de orden dos \ref{ec:diferencias_centradas}.

\ptitle{Obtención del sistema lineal en el borde derecho}
Análogamente, se puede repetir el procedimiento para el nodo de borde $i = N$. Planteando
\[y_N'' + b_{N-1} y_{N-1}'' = a_{N-1} y_{N-1} + a_N y_N + a_{N+1} y_{N+1} +O(h^\alpha)\]
y desarrollando en Taylor $y_{N-1}''$, $y_{N-1}$ e $y_{N+1}$ alrededor de $x_N$, se obtiene un sistema de ecuaciones para los coeficientes. La aproximación de Padé resultante es de segundo orden y su expresión es
\begin{equation}
    y_N'' = \frac{y_{N+1} - 2 y_N + y_{N-1}}{h^2} + O(h^2),
    \label{eq:Pade_der}
\end{equation}
idéntica a la de diferencias centradas de orden dos \ref{ec:diferencias_centradas}.

\ptitle{Obtención del sistema lineal}
Empleando las fórmulas de diferencia finita de Padé de segundo orden (\ref{eq:Pade_izq}, \ref{eq:Pade_der} y \ref{ec:Pade_interior}) se puede escribir la relación lineal $A_P \vec{y''} = B_P \vec{y} + \vec{c}$, donde
\begin{equation}
    A_P = \left(\begin{matrix}
    1 & 0 & 0 & \cdots & 0 & 0 & 0 \\
    \frac{1}{12} & \frac{10}{12} & \frac{1}{12} & \cdots & 0 & 0 & 0 \\
    0 & \frac{1}{12} & \frac{10}{12} & \cdots & 0 & 0 & 0 \\
    \vdots & \vdots & \vdots & \ddots & \vdots & \vdots & \vdots \\
    0 & 0 & 0 & \cdots & \frac{10}{12} & \frac{1}{12} & 0 \\
    0 & 0 & 0 & \cdots & \frac{1}{12} & \frac{10}{12} & \frac{1}{12} \\
    0 & 0 & 0 & \cdots & 0 & 0 & 1
    \end{matrix}\right)
\end{equation}

\begin{equation}
    B_P = \frac{1}{h^2} \left(\begin{matrix}
    -2 & 1 & 0 & \cdots & 0 & 0 & 0 \\
    1 & -2 & 1 & \cdots & 0 & 0 & 0 \\
    0 & 1 & -2 & \cdots & 0 & 0 & 0 \\
    \vdots & \vdots & \vdots & \ddots & \vdots & \vdots & \vdots \\
    0 & 0 & 0 & \cdots & -2 & 1 & 0 \\
    0 & 0 & 0 & \cdots & 1 & -2 & 1 \\
    0 & 0 & 0 & \cdots & 0 & 1 & -2
    \end{matrix}\right)
\end{equation}

ambas matrices tridiagonales. Mientras que
\[\vec{y''} = (y_1'', y_2'', \cdots, y_N''),\]
\[\vec{y} = (y_1, y_2, \cdots, y_N)\]
y
\[\vec{c} = (y(0)/h^2, 0, \cdots, 0, y(1)/h^2)\]

Aplicando esta relación lineal a la ecuación diferencial \ref{ec:PVC_numerico} se obtiene
\[(B_P - A_P)\vec{y} = A_P\vec{g} - \vec{c}\]
donde $\vec{g} = (g_1, g_2, \cdots, g_N)$. 


\ptitle{Resumen de los que se tratará en Resultados}
Habiendo desarrollado ambos métodos y habiéndolos aplicado a la ecuación diferencial, se está en condiciones de resolver el problema y comparar con la solución exacta.


\section{Resultados}
\textcolor{red}{Borrar: En primer lugar, se analizó cualitativamente ambas aproximaciones para $K$ y $N$ fijos. En segundo lugar, se estudió la dependencia del error local en un punto particular en función del espaciamiento $h$. Finalmente, se estudió cuántos puntos discretos $N$ son necesarios para obtener un determinado error global en función de las oscilaciones de la solución representadas por el parámetro $K$.}

\ptitle{Comportamiento cualitativo}
Se analizó cualitativamente ambas aproximaciones para $K$ y $N$ fijos. En la figura \ref{fig:comportamiento_cuali} se grafica la solución exacta junto a las soluciones aproximadas en puntos discretos por ambos métodos.
\textcolor{red}{¿Qué se boserva?}

\begin{figure}[h]
    \includegraphics[clip=true,width=\columnwidth]{nada.png}
    \caption{Solución exacta y soluciones aproximadas en puntos discretos por los métodos de diferencias finitas centradas y aproximación de Padé para $K = 6$ y $N = 23$.}
     \label{fig:comportamiento_cuali}
\end{figure}

\ptitle{Error en x = 0.5}
Además, se calculó el error local en el punto central $x = 0.5$ en función del espaciamiento $h$ para ambos métodos numéricos. Esta dependencia se grafica en la figura \ref{fig:error_local_0.5} en escala log-log.


\begin{itemize}
    \item Discutir la dependencia en cada zona: para h grande el error es grande. A medida que h disminuye se encuentra un comportamiento polinómico. Luego de este régimen intermedio el error vuelve a aumentar como consecuencia del error de precisión de la computadora.
    \item En base a la dependecia observada en la zona intermedia se ajustó una función del tipo $x^\beta$ para ambas, donde $\beta$ sería el orden del error de truncamiento en ese punto
    \item Se obtuvo para el método de diferencias finitas centradas \textcolor{red}{$\beta = 2$} y para el método de Padé \textcolor{red}{$\beta = 4$}.
    \item Es consistente que el método de Padé tenga mayor orden que el de diferencias centradas debido a que el error local en los puntos intermedios es mayor, como se verifica al comparar \ref{ec:diferencias_centradas} y \ref{ec:Pade_interior}.
    \item Sin embargo, como el orden de aproximación en los bordes en ambos casos es 2, se esperaría que en Padé el orden local esté entre 2 y 4. Habría que estudiar esto con mayor detalle.
\end{itemize}

\begin{figure}[h]
    \includegraphics[clip=true,width=\columnwidth]{nada.png}
    \caption{Error local en el punto $x = 0.5$ en función del espaciamiento $h$ para diferencias finitas centradas y aproximación de Padé en escala log-log. Se ajustó en ambos casos una dependencia $x^\beta$ correspondiente a una línea de pendiente $\beta$ en escala log-log. En el caso de diferencias finitas se obtuvo una pendiente $\beta_{DF} = $ \textcolor{red}{cuánto?}. Mientras que en la aproximación de Padé, $\beta_P = $ \textcolor{red}{cuánto?} }
     \label{fig:error_local_0.5}
\end{figure}

\ptitle{Se determinó el nro de puntos mínimos necesarios para obtener un error menor a $10e-1$ en distintas normas en función de la frecuencia K}

\begin{itemize}
    \item Se determinó el nro de puntos mínimos necesarios para obtener un error menor a 10e-1 en distintas normas en función de la frecuencia K
    \item \ref{fig:NvsK}
    \item Se obtuvo que a mayor $K$, es decir, oscilaciones de mayor frecuencia en la solución exacta, más puntos se necesitan para disminuir el error global.
\end{itemize}

\begin{figure}[h]
    \includegraphics[clip=true,width=\columnwidth]{nada.png}
    \caption{\textcolor{blue}{N tal que el error es menor a $10e-1$ en función de K apra distintas normas}.}
     \label{fig:NvsK}
\end{figure}


\ptitle{Error en el punto central en función de h}


\begin{itemize}
    \item test
\end{itemize} 

\section{Conclusión}
\begin{itemize}
    \item test
\end{itemize}


\bibliography{Chehade_guia1.bib}

\end{document}





