\documentclass[aps,prb,twocolumn,superscriptaddress,floatfix,longbibliography]{revtex4-2}

\usepackage[utf8]{inputenc}
\usepackage[spanish]{babel}
\usepackage{graphicx}
\usepackage{amsmath}
\usepackage{subcaption}
\usepackage{wrapfig} 
\usepackage[export]{adjustbox}

\usepackage{amsmath,amssymb} % math symbols
\usepackage{bm} % bold math font
\usepackage{graphicx} % for figures
\usepackage{comment} % allows block comments
\usepackage{textcomp} % This package is just to give the text quote '
%\usepackage{ulem} % allows strikeout text, e.g. \sout{text}

\usepackage[spanish]{babel}

\usepackage{enumitem}
\setlist{noitemsep,leftmargin=*,topsep=0pt,parsep=0pt}

\usepackage{xcolor} % \textcolor{red}{text} will be red for notes
\definecolor{lightgray}{gray}{0.6}
\definecolor{medgray}{gray}{0.4}

\usepackage{hyperref}
\hypersetup{
colorlinks=true,
urlcolor= blue,
citecolor=blue,
linkcolor= blue,
bookmarks=true,
bookmarksopen=false,
}

% Code to add paragraph numbers and titles
\newif\ifptitle
\newif\ifpnumber
\newcounter{para}
\newcommand\ptitle[1]{\par\refstepcounter{para}
{\ifpnumber{\noindent\textcolor{lightgray}{\textbf{\thepara}}\indent}\fi}
{\ifptitle{\textbf{[{#1}]}}\fi}}
%\ptitletrue  % comment this line to hide paragraph titles
%\pnumbertrue  % comment this line to hide paragraph numbers

% minimum font size for figures
\newcommand{\minfont}{6}

% Uncomment this line if you prefer your vectors to appear as bold letters.
% By default they will appear with arrows over them.
% \renewcommand{\vec}[1]{\bm{#1}}

%Cambiar Cuadros por Tablas y lista de...
%\renewcommand{\listtablename}{Índice de tablas}
\renewcommand{\tablename}{Tabla}
\renewcommand{\date}{Fecha}

\graphicspath{ {C:/Users/lupam/Mi unidad/Pablo Chehade/Instituto Balseiro (IB)/Métodos numéricos en fluidos I/Metodos_Num_Fluidos_I/Guias/Guia_1/Informe/Figures} } %Para importar imágenes desde una carpeta


\usepackage[bottom]{footmisc} %para que las notas al pie aparezcan en la misma página



\begin{comment}

%Comandos de interés:

* Para ordenar el documento:
\section{Introducción}
\section{\label{sec:Formatting}Formatting} %label para luego hacer referencia a esa sección

\ptitle{Start writing while you experiment} %pone nombre y título al documento dependiendo de si en el header están los comandos \ptitletrue y \pnumbertrue

* Ecuaciones:
\begin{equation}
a^2+b^2=c^2 \,.
\label{eqn:Pythagoras}
\end{equation}

* Conjunto de ecuaciones:
\begin{eqnarray}
\label{eqn:diagonal}
\nonumber d & = & \sqrt{a^2 + b^2 + c^2} \\
& = & \sqrt{3^2+4^2+12^2} = 13
\end{eqnarray}

* Para hacer items / enumerar:
\begin{enumerate}
  \item
\end{enumerate}

\begin{itemize}
  \item
\end{itemize}

* Figuras:
\begin{figure}[h]
    \includegraphics[clip=true,width=\columnwidth]{pixel-compare}
    \caption{}
     \label{fig:pixels}
\end{figure}

* Conjunto de figuras:
(no recuerdo)


* Para hacer referencias a fórmulas, tablas, secciones, ... dentro del documento:
\ref{tab:spacing}

* Para citar
Elementos de .bib
\cite{WhitesidesAdvMat2004}
url
\url{http://www.mendeley.com/}\\

* Agradecimientos:
\begin{acknowledgments}
We acknowledge advice from Jessie Zhang and Harry Pirie to produce Fig.\ \ref{fig:pixels}.
\end{acknowledgments}

* Apéndice:
\appendix
\section{\label{app:Mendeley}Mendeley}

* Bibliografía:
\bibliography{Hoffman-example-paper}

\end{comment}



\begin{document}

% Allows to rewrite the same title in the supplement
\newcommand{\mytitle}{Laboratorio 1}

\title{\mytitle}

\author{Pablo Chehade \\
    \small \textit{pablo.chehade@ib.edu.ar} \\
    \small \textit{Métodos Numéricos en Fluidos I, Instituto Balseiro, CNEA-UNCuyo, Bariloche, Argentina, 2022} \\}


\begin{abstract}

Se estudió el comportamiento de distintas aproximaciones numéricas en un problema de valores de contorno tipo Dirichlet de solución exacta conocida.


Se obtuvo...

\end{abstract}

\maketitle

\textcolor{red}{
\begin{itemize}
    \item ¿Tengo que mencionar la computadora que usé? ¿En qué sección lo hago?
    \item ¿Está bien el nombre de la sección "Método Numérico"?
    \item ¿Cuánto desarrollo hay que hacer en b y c?
    \item ¿Es necesario hacer el desarrollo para obtener la solución exacta o se puede dar por sabido?
\end{itemize}}

\section{Introducción}
\ptitle{En física, no todos los problemas tienen solución analítica, muchas veces es necesario recurrir a aproximaciones}
\begin{itemize}
    \item Estaría bueno mencionar ejemplos de ecuaciones diferenciales lineales que sólo se pueden atacar de forma numérica. No encontré ningún ejemplo
\end{itemize}
    
\end{array}

\ptitle{En este trabajo se estudió el comportamiento de distintas aproximaciones numéricas en el siguiente problema de valores de contorno tipo Dirichlet}
\begin{itemize}
    \item Describir la ec
    \item Este problema tiene solución exacta conocida. Describir la solución exacta y referenciar a un anexo para la discusión de cómo se encontró la solución
\end{itemize}

\section{Método Numérico}

\ptitle{Discretización del dominio}
\begin{itemize}
    \item Discretización del dominio
\end{itemize}

El dominio se discretiza con puntos xi = ih, i = 1, . . . , N , y h = 1/(N + 1). Notar que no hay puntos en los contornos. En este problema yi es la estimacion de y en el punto xi, y $\mathrm{y_{vec}}$ = (y1, y2, ..., yN )T es el vector soluci ́on

\ptitle{Para estimar y'' emplearon diferencias finitas centradas de segundo orden y el \textcolor{red}{esquema} de Padé \textcolor{red}{de 4to orden}}
\textcolor{red}{Parafrasear: } El valor de y'' se puede estimar como combinaci ́on  lineal de los valores de y empleando diferencias finitas, transformando el problema de EDO en  un sistema de ecuaciones lineales.

\subsection{Diferencias finitas centradas}

\ptitle{Obtención del sistema lineal}
\begin{itemize}
    \item Ec del Moine para dif centradas (referencia al Moin)
    \item Aplicación de la aproximación a la EDO. Poner directamente la ec. final
    \item Esta aproximación vale también en los puntos del borde, bajo la consideración de que $y_0 = y(0)$ e $y_{N+1} = y(1)$. Presentación de las ecuaciones
    \item El sistema de ecuaciones algebraicas anterior se puede describir a través de la ec. A1y_vec = b1_vec, donde
\end{itemize}


\subsection{Padé}

\ptitle{Obtención del sistema lineal en los nodos internos}

\begin{itemize}
    \item Presentación de la aprox (referenciada al Moin) junto a la región de validez sobre i.
\end{itemize}

\ptitle{Obtención del sistema lineal en el borde izquierdo}
\begin{itemize}
    \item Para los nodos del borde se puede deribar una aproximación de Padé para y1'' e yN''. Para esto basta plantear \textcolor{blue}{ec. para encontrar la aprox de Padé en el nodo izq} y determinar los coeficientes a0, a1, a2 y b2, de modo de obtener el mayor orden de aproximación posible. Para esto, se desarrolla en Taylor y2, y2'' e y0 alrededor de \textcolor{red}{x1?}.
    \item \textcolor{blue}{Ec y1'' = y1(...)....}
    \item Igualando ambos miembros se obtienen los coeficientes \textcolor{blue}{Coefs}
\end{itemize}

\ptitle{Obtención del sistema lineal en el borde derecho}
Análogo al caso anterior, presentar las ecuaciones y las soluciones

\ptitle{Obtención del sistema lineal}
\begin{itemize}
    \item Empleando las fórmulas de Padé para los nodos internos y los bordes, se obtiene la relación lineal A2y_vec'' = B2 y_vec + c, donde
    \item Aplicandola a la ecuación diferencial (ref) se obtiene
    \item \textcolor{blue}{ec. encuadrada luego de la presentación de la relación lineal}
\end{itemize}

\ptitle{Resumen de los que se tratará en Resultados}

\section{Resultados}

\ptitle{Comportamiento cualitativo}
En primer lugar, se observó cualitativamente el comportamiento de los esquemas numéricos para K = 6 y N = 23

\ptitle{Error en el punto central en función de h}




\begin{itemize}
    \item
\end{itemize}

\section{Conclusión}
\begin{itemize}
    \item
\end{itemize}

\section{Anexo}

\subsection{Solución exacta del problema de valores de contorno}
\begin{itemize}
    \item La solución general se puede expresar como una solución de la ec homogéna
    \item \textcolor{blue}{ec. homogénea}
    \item y una solución particular
    \item Obtención de la solución homogénea (brevemente, sin hacer muchas cuentas)
    \item Obtención de la solución particular
    \item Obtención de las ctes C1 y C2
    \item Solución gral
\end{itemize}

\bibliography{Chehade_guia1.bib}

\end{document}





