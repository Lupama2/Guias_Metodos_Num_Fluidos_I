\documentclass[aps,prb,twocolumn,superscriptaddress,floatfix,longbibliography,10pt]{revtex4-2}

\usepackage[utf8]{inputenc}
\usepackage[spanish]{babel}
\usepackage{graphicx}
\usepackage{amsmath}
\usepackage{subcaption}
\usepackage{wrapfig} 
\usepackage[export]{adjustbox}

\usepackage{amsmath,amssymb} % math symbols
\usepackage{bm} % bold math font
\usepackage{graphicx} % for figures
\usepackage{comment} % allows block comments
\usepackage{textcomp} % This package is just to give the text quote '
%\usepackage{ulem} % allows strikeout text, e.g. \sout{text}

\usepackage[spanish]{babel}
% By dafault, spanish changes to a comma as decimal separator; to change to a dot, you can use \decimalpoint:
\decimalpoint

\usepackage{enumitem}
\setlist{noitemsep,leftmargin=*,topsep=0pt,parsep=0pt}

\usepackage{xcolor} % \textcolor{red}{text} will be red for notes
\definecolor{lightgray}{gray}{0.6}
\definecolor{medgray}{gray}{0.4}

%Para las tablas
\usepackage{multirow}

\usepackage{hyperref}
\hypersetup{
colorlinks=true,
urlcolor= blue,
citecolor=blue,
linkcolor= blue,
bookmarks=true,
bookmarksopen=false,
}

% Code to add paragraph numbers and titles
\newif\ifptitle
\newif\ifpnumber
\newcounter{para}
\newcommand\ptitle[1]{\par\refstepcounter{para}
{\ifpnumber{\noindent\textcolor{lightgray}{\textbf{\thepara}}\indent}\fi}
{\ifptitle{\textbf{[{#1}]}}\fi}}
% \ptitletrue  % comment this line to hide paragraph titles
% \pnumbertrue  % comment this line to hide paragraph numbers

% minimum font size for figures
\newcommand{\minfont}{6}

% Uncomment this line if you prefer your vectors to appear as bold letters.
% By default they will appear with arrows over them.
% \renewcommand{\vec}[1]{\bm{#1}}

%Cambiar Cuadros por Tablas y lista de...
%\renewcommand{\listtablename}{Índice de tablas}
\renewcommand{\tablename}{Tabla}
\renewcommand{\date}{Fecha}

%Para importar imágenes desde una carpeta:
\graphicspath{ {C:/Users/lupam/OneDrive/Escritorio/GitHub/Metodos_Num_Fluidos_I/Guias/Informe_final/Informe/Figures} {C:/Users/lupam/OneDrive/Escritorio/GitHub/Metodos_Num_Fluidos_I/Guias/Informe_final/Programa/graficos}}

\usepackage[bottom]{footmisc} %para que las notas al pie aparezcan en la misma página

\begin{comment}

%Comandos de interés:

* Para ordenar el documento:
\section{Introducción}
\section{\label{sec:Formatting}Formatting} %label para luego hacer referencia a esa sección

\ptitle{Start writing while you experiment} %pone nombre y título al documento dependiendo de si en el header están los comandos \ptitletrue y \pnumbertrue

* Ecuaciones:
\begin{equation}
a^2+b^2=c^2 \,.
\label{eqn:Pythagoras}
\end{equation}

* Conjunto de ecuaciones:
\begin{eqnarray}
\label{eqn:diagonal}
\nonumber d & = & \sqrt{a^2 + b^2 + c^2} \\
& = & \sqrt{3^2+4^2+12^2} = 13
\end{eqnarray}

* Para hacer items / enumerar:
\begin{enumerate}
  \item
\end{enumerate}

\begin{itemize}
  \item
\end{itemize}

* Figuras:
\begin{figure}[h]
    \includegraphics[clip=true,width=\columnwidth]{pixel-compare}
    \caption{}
     \label{fig:pixels}
\end{figure}

* Conjunto de figuras:
\begin{figure}
     \centering
     \begin{subfigure}[b]{0.3\textwidth}
         \centering
         \includegraphics[width=\textwidth]{graph1}
         \caption{$y=x$}
         \label{fig:y equals x}
     \end{subfigure}
     \hfill
     \begin{subfigure}[b]{0.3\textwidth}
         \centering
         \includegraphics[width=\textwidth]{graph2}
         \caption{$y=3sinx$}
         \label{fig:three sin x}
     \end{subfigure}
     \hfill
     \begin{subfigure}[b]{0.3\textwidth}
         \centering
         \includegraphics[width=\textwidth]{graph3}
         \caption{$y=5/x$}
         \label{fig:five over x}
     \end{subfigure}
        \caption{Three simple graphs}
        \label{fig:three graphs}
\end{figure}


* Para hacer referencias a fórmulas, tablas, secciones, ... dentro del documento:
\ref{tab:spacing}

* Para citar
Elementos de .bib
\cite{WhitesidesAdvMat2004}
url
\url{http://www.mendeley.com/}\\

* Agradecimientos:
\begin{acknowledgments}
We acknowledge advice from Jessie Zhang and Harry Pirie to produce Fig.\ \ref{fig:pixels}.
\end{acknowledgments}

* Apéndice:
\appendix
\section{\label{app:Mendeley}Mendeley}

* Bibliografía:
\bibliography{Hoffman-example-paper}

\end{comment}



\begin{document}

% Allows to rewrite the same title in the supplement
\newcommand{\mytitle}{Métodos numéricos en Fluidos}

\title{\mytitle}

\author{Pablo Chehade \\
    \small \textit{pablo.chehade@ib.edu.ar} \\
    \small \textit{Métodos Numéricos en Fluidos I, Instituto Balseiro, CNEA-UNCuyo, Bariloche, Argentina, 2022} \\}


\begin{abstract}

Se resolvió el problema de la cavidad cuadrada hidrodinámica bidimensional mediante el método de volumenes finitos y algoritmo SIMPLER calculando presión y velocidades con grilla desplazada. Debido a la cantidad de variables involucradas, la resolución conlleva un gran costo computacional. Se plantearon algoritmos que permiten minimizar el tiempo de cómputo, a costo de aumentar el error de la solución. Estos se basan en aumentar el paso de tiempo y la discretización espacial de forma controlada. Además, se estudiaron distintos métodos numéricos espaciales y de evolución temporal. En particular, se analizaron los esquemas de diferencias finitas de orden 2, Upwind de orden 1 y el esquema QUICK para el término advectivo de la ecuación de momento. Mientras que para la evolución temporal, los métodos Euler implícito y Crank-Nicholson. Se estudió también la dependencia del estado estacionario con el paso de tiempo, el tamaño de la discretización espacial y el número de iteraciones internas en el algoritmo SIMPLER.



\end{abstract}

\maketitle

\section{Introducción}

\ptitle{¿Por qué es importante resolver problemas de fluidos numéricamente? Rtas en la primera clase}

Gran parte de los problemas de mecánica de fluidos no son resolubles analíticamente. Algunos de ellos pueden ser estudiados experimentalmente, con el costo operativo y las dificultades para realizar las mediciones que esto conlleva \cite{Notas_materia}. Una alternativa más rápida y de menor costo es resolverlos numéricamente. Si bien esto trae aparejado algunas dificultades, como los errores de aproximación numérica y el costo computacional, en los últimos años la resolución numérica de ecuaciones es aceptada y está ganando preponderancia.

\ptitle{Explicar el problema de la cavidad cuadrada hidrodinámica bidimensional}
Un problema muy estudiado desde el punto de vista numérico en mecánica de fluidos es el de la cavidad cuadrada hidrodinámica bidimensional, también conocido como Shear-drien cavity flow. En la figura \ref{fig:esquema_cavidad_cuadrada} se muestra un esquema del mismo. Consiste en una cavidad cuadrada bidimensional de lado $L$ que contiene un fluido incompresible de viscocidad $\nu$. La condición inicial es de velocidad nula en todo el dominio y las condiciones de borde son de no deslizamiento en todas las paredes excepto en la horizontal superior. Sea $\vec{V}(x,y,t) = (u(x,y,t), v(x,y,t))$ la velocidad en el punto $(x,y)$ a tiempo $t$, las condiciones anteriores se traducen en $\vec{V}(x,0) = \vec{V}(0,y) = \vec{V}(L,y) = (0,0)$ y $\vec{V}(x,L) = (U_0(t),0)$.

\begin{figure}[h]
  \includegraphics[clip=true,width=0.7\columnwidth]{esquema_cavidad_cuadrada.png}
  \caption{Esquema del problema de la cavidad cuadrada hidrodinámica bidimensional. Consiste en una cavidad cuadrada bidimensional de lado $L$ que contiene un fluido incompresible de viscocidad $\nu$. Las condiciones de borde son de no deslizamiento en todas las paredes excepto en la horizontal superior, con velocidad $U_0$. Esta figura fue extraída de \cite{Notas_materia}.}
   \label{fig:esquema_cavidad_cuadrada}
\end{figure}



\ptitle{Explicación de las ecuaciones involucradas}

El objetivo es determinar la presión $p(x,y,t)$ y las velocidades $u(x,y,t)$ y $v(x,y,t)$ para todo punto $(x,y)$ y tiempo $t$. Para esto se cuenta con tres ecuaciones diferenciales. En primer lugar, la ecuación de conservación de momento en las direcciones vertical y horizontal. Adimensionalmente, estas son
\begin{equation}
  \frac{\partial u}{\partial t} + \frac{\partial (u u)}{\partial x} + \frac{\partial (u v)}{\partial y} = - \frac{\partial p}{\partial x} + \frac{1}{Re} \left ( \frac{\partial^2 u}{\partial x^2} + \frac{\partial^2 u}{\partial y^2} \right ),
  \label{eq:momento_x}
\end{equation}
\begin{equation}
  \frac{\partial v}{\partial t} + \frac{\partial (u v)}{\partial x} + \frac{\partial (v v)}{\partial y} = - \frac{\partial p}{\partial y} + \frac{1}{Re} \left ( \frac{\partial^2 v}{\partial x^2} + \frac{\partial^2 v}{\partial y^2} \right ),
  \label{eq:momento_y}
\end{equation}
donde $Re = U_0L/\nu$ es el número de Reynolds. En ambas ecuaciones, el primer término del lado izquierdo resume la dependencia temporal del problema, los términos siguientes corresponden al efecto advectivo, el primer término del lado derecho resume la dependencia con la presión y los siguientes, el efecto difusivo. En segundo lugar se cuenta con la ecuación de conservación de masa. Para un fluido incompresible esta es
\begin{equation}
  \nabla \bullet \vec{V} = \frac{\partial u}{\partial x} + \frac{\partial v}{\partial y} = 0.
  \label{eq:cons_masa}
\end{equation}
En base a esta ecuación se suele decir que el campo de velocidades tiene divergencia libre.

\ptitle{Resumen}

Empleando las tres ecuaciones diferenciales en derivadas parciales anteriores, es posible calcular la presión y las velocidades. Es notable que las soluciones dependerán del valor del número de Reynolds, el cual depende de la velocidad en la cara superior, la longitud de la cavidad y la viscocidad del líquido.

\section{Métodos Numéricos}

\ptitle{Resumen}
Para resolver el sistema de ecuaciones diferenciales es necesario discretizar el dominio y plantear un esquema numérico para las ecuaciones diferenciales. En cuanto al primero, es necesario diferenciar entre dominio espacial y temporal. En cuanto al segundo, se emplearon volúmenes finitos y distintos métodos espaciales y temporales.

\subsection{Discretización del dominio}

\ptitle{Discretización espacial}
El dominio espacial se discretizó mediante una grilla uniforme con espaciamiento $\Delta$ en ambas direcciones. El número de volúmenes por dirección espacial es $n$ tal que $\Delta = 1/n$. En total se cuentan con $n^2$ volúmenes, cada uno numerado con los índices $i$ y $j$ para las direcciones horizontales y verticales, respecticamente.

\ptitle{Grilla desplazada}
Se emplearon grillas desplazadas para las velocidades y la presión. Estas se muestran esquemáticamente en la figura \ref{fig:grillas_desplazadas}. De este modo, $P_{ij}$ es la presión en el volumen de tamaño $\Delta \times \Delta$ ubicado en $x = i \Delta$ e $y = j \Delta$. Mientras que $U_{ij}$ es la velocidad horizontal en $x = (i+1/2)\Delta $ e $y = j\Delta$ y $V_{ij}$, la velocidad vertical en $x = i \Delta$ e $y = (j + 1/2) \Delta$. En base a lo anterior, la grilla para $p$ posee todos sus volúmenes contenidos en la cavidad. Como no se necesita conocer la presión en el borde del domindio, no es necesario describir sus condiciones de borde, las cuales suelen ser difíciles de determinar. Esta justamente es la ventaja de emplear grillas desplazadas.

\begin{figure}
  \centering
  \begin{subfigure}[b]{0.25\textwidth}
      \includegraphics[width=\textwidth]{grilla_diferida_u.png}
      \caption{}
      \label{fig:grilla_desplazada_u}
  \end{subfigure}
  \hfill
  \begin{subfigure}[b]{0.2\textwidth}
      \includegraphics[width=\textwidth]{grilla_diferida_v}
      \caption{}
      \label{fig:grilla_desplazada_v}
  \end{subfigure}
     \caption{Esquema de las grillas desplazadas para la presión $p$ y las velocidades horizontal $u$ (\ref{fig:grilla_desplazada_u}) y vertical $v$ (\ref{fig:grilla_desplazada_v}). La grilla para $p$ es la única que posee todos sus volúmenes contenidos en la cavidad. Estas figuras fueron extraídas de \cite{Notas_materia}.}
     \label{fig:grillas_desplazadas}
\end{figure}


\ptitle{Discretización temporal}
Por otro lado, el dominio temporal se define desde $t = 0$ hasta $t = t_{max}$. El tiempo se discretizó en puntos equiespaciados $t_k = k \Delta t$ con $k = 0,1,\dots,M$ con $\Delta t = t_{max}/M$. De este modo, la presión $P_{ij}^k$ es la $P_{ij}$ evaluada en $t = t_k$ y de forma análoga con las demás variables.


\subsection{Esquema numérico}

\ptitle{Resumen}
Se construyó un esquema numérico para las ecuaciones de conservación de momento \ref{eq:momento_x} y \ref{eq:momento_y} mediante la técnica de volúmenes finitos. Para aproximar los términos obtenidos se emplearon distintos métodos numéricos. Para la evolución temporal se utilizó Euler Implícito y Crank-Nicolson. Mientras que para el término advectivo se empleó diferencias finitas centradas de orden 2, Up-wind de orden 1 y el esquema QUICK \cite{Hayase1992}. Por último, para el difusivo se empleó diferencias centradas de orden 2. Una vez planteado el esquema numérico, se resolvió el sistema de ecuaciones mediante el algoritmo SIMPLER exigiendo que el campo de velocidades tenga divergencia libre en base a \ref{eq:cons_masa}. A continuación se explica la construcción del esquema numérico, junto a las características particulares de cada método utilzado.

\ptitle{Volúmenes finitos}
La implementación de la técnica de volúmenes finitos para este problema en particular se explica en detalle en \cite{Notas_materia}. De forma resumida, se integran las ecuaciones de momento \ref{eq:momento_x} y \ref{eq:momento_y} en el volumen de las grillas de $u$ y $v$, respecticamente, empleando la regla del rectángulo \cite{Moin}. Esta aproximación es de orden mayor a $\Delta^2$, correspondiente al mayor orden de aproximación de los esquemas para los términos advectivo y difusivo. De este modo se arriba a un sistema de ecuaciones donde el término temporal depende de $\partial U_{ij}/ \partial t$, el advectivo depende de $u$ evaluada en los cuatro laterales: $U_{N,ij}$ norte, $U_{S,ij}$ sur, $U_{E,ij}$ este y $U_{W,ij}$ oeste, y de $v$ evaluada de igual manera. Una particularidad del término advectivo es su no linealidad, que se resuelve empleando como uno de los factores la solución del paso previo o de iteraciones internas. Además, el término de presión depende de $P_{ij}$ y el término difusivo depende de las derivadas espaciales de $u$ evaluadas en los laterales: $\frac{\partial U_{E,ij}}{\partial x}$, $\frac{\partial U_{W,ij}}{\partial x}$, $\frac{\partial U_{N,ij}}{\partial y}$ y $\frac{\partial U_{S,ij}}{\partial y}$, y de las derivadas espaciales de $v$ de forma análoga. Es necesario aclarar que las variables anteriores se evalúan tanto en los laterales de los volúmenes de la grilla de $u$ como en los de $v$. Sin embargo, las aproximaciones para ambos casos son análogas. En base a las dependencias presentadas, aplicar un método numérico determinado implica aproximar las velocidades $U_{ij}$, $V_{ij}$ y sus derivadas en el centro y los laterales de sus volúmenes finitos respectivos. Además, por simetría $U_{E,ij} = U_{W,i+1j}$, $U_{N,ij} = U_{S,ij+1}$ y cualquier aproximación sobre $U_{W,ij}$ vale sobre $U_{S,ij}$ intercambiando los índices $ij$ entre sí. De este modo, solo basta describir la aproximación en uno de los laterales


\ptitle{Métodos espaciales: presentación y aproximaciones para el término advectivo}
A continuación se desarrollan los métodos numéricos utilizados para el dominio espacial. Sólo se presentan las expresiones para la velocidad horizontal $u$, considerando que para la velocidad vertical $v$ son análogas. En cuanto al término advectivo, se empleron tres métodos numéricos: diferencias centradas de orden 2 (DC1), Up-wind de orden 1 (UP1) y el esquema QUICK \cite{Hayase1992}. En diferencias centradas de orden 2 se aproxima
\[U_{W,ij} = \frac{U_{ij} + U_{i-1j}}{2}. \]
Este método tiene la desventaja de presentar oscilaciones espúreas. Por otro lado, en Up-wind de orden 1 se aproxima
\[U_{W,ij} =  \left\{\begin{matrix}
  U_{i-1j} \qquad \mathrm{si} \qquad U_{W,ij}^*\geqslant 0\\
  U_{ij} \quad \qquad \mathrm{si} \qquad U_{W,ij}^* < 0
  \end{matrix}\right.\]
donde el subíndice '*' indica que corresponde al paso previo o a iteraciones internas. Tiene la ventaja de no presentar oscilaciones espúreas pero las desventajas de ser difusivo y de menor orden que DC2. Por último, en el esquema QUICK se aproxima
\[U_{W,ij} =  \left\{\begin{matrix}
  U_{i-1j} + \frac{1}{8}( - U_{i-2j} -2 U_{i-1j} + 3U_{ij} ) \, \mathrm{si} U_{W,ij}^*\geqslant 0\\
  U_{ij} + \frac{1}{8}(3 U_{i-1j} - 2 U_{ij} - U_{i+1j}) \quad \quad \mathrm{si} U_{W,ij}^* < 0
\end{matrix}\right.\]
escrito como UP1 más términos de fuente.

\ptitle{Aproximaciones para el término difusivo}
En cuanto al término difusivo, se emplearon diferencias finitas centradas de orden 2:
\[\frac{\partial U_{W,ij}}{\partial t} =  \frac{U_{ij} - U_{i-1j}}{\Delta}\]


\ptitle{Método de evolución temporal}
En cuanto a la evolución temporal, el sistema de ecuaciones discretizadas se puede llevar a uno de la forma
\[\frac{\partial U_{ij}^k}{\partial t} = f(U^k,V^k,P^k)\]
\[\frac{\partial V_{ij}^k}{\partial t} = g(U^k,V^k,P^k)\]
donde $f$ y $g$ son funciones y $U^k$, $V^k$ y $P^k$ son las matrices de velocidad y presión evaluadas a tiempo $t = t_k$. En ambos casos la derivada temporal se aproximó mediante los métodos Euler Implícito (EI) y Crank-Nicholson (CN). En el primero, se aproximó
\[U_{ij}^{k+1} = U_{ij}^{k} + \Delta t f(U^{k+1},V^{k+1},P^{k+1})  + O_{local}({\Delta t}^2)\]
y de forma análoga para $\partial V_{ij}^k / \partial t$. En el segundo, se aproximó
\[
\begin{split}  
U_{ij}^{k+1} = U_{ij}^{k} + \frac{\Delta t}{2} [f(U^{k+1},V^{k+1},P^{k+1}) + f(U^{k},V^{k},P^{k})] + \\ O_{local}({\Delta t}^3).
\end{split}\]
y de forma análoga para $\partial V_{ij}^k / \partial t$. Este método implícito tiene la ventaja de tener factor de amplificación unitario y de ser de mayor orden que EI


\ptitle{Algoritmo simpler simplificadamente}
¿Mediante las aproximaciones anteriores, las ecuaciones de conservación de momento \ref{eq:momento_x} y \ref{eq:momento_y} se transforman en dos ecuaciones que sólo dependen de $U^k$, $U^{k-1}$, $V^k$, $V^{k-1}$, $P^k$ y $P^{k-1}$. Estas ecuaciones junto a la condición de divergencia libre \ref{eq:cons_masa} se resuelven mediante el algoritmo SIMPLER. \cite{Patankar}. Este es un algoritmo segregado o de paso fraccionado en el que se calculan por separado velocidad y presión. Partiendo de una aproximación de la velocidad y presión a tiempo $t_k$ es capaz de mejorarla mediante un método iterativo. Tal aproximación inicial suele ser la solución numérica a tiempo $t_{k-1}$. La exactitud de la solución depende del número de iteraciones internas $l_{simpler}$ realizadas, a expensas de un mayor costo computacional.



\subsection{Costo computacional}

\ptitle{Costo computacional y elección de $n$}
Debido a la cantidad de variables involucradas proporcional a $n^2$ y dependiendo del número de pasos de tiempo a realizar, el costo computacional puede ser alto. En base a esto, es importante desarrollar métodos que permitan disminuir el costo pero que a su vez tengan una precisión aceptable. Un modo de minimizarlo es disminuyendo el número de variables pero exigiendo a su vez un error menor al 5 \% respecto a la solución exacta, por ejemplo. Sin embargo, para esto es necesario conocer tal solución o al menos una muy buena aproximación, lo cual no siempre se tiene. Debido a que el problema analizado está ampliamente estudiado en la bibliografía, se cuenta con una solución numérica de gran precisión \cite{Guia1982}. Para disminuir $n$ se puede emplear el siguiente algoritmo: se parte de $n_{guess}$ lo suficientemente grande para asegurar un error menor al 5 \%. Luego, se reduce $n$ en dos unidades, se avanza la solución y se calcula el error en el estado estacionario. Si este es menor al 5 \% se disminuye nuevamente $n$ y se repite el procedimiento. Si es mayor al 5 \%, se define como el mejor valor de $n$ al valor anterior.

\ptitle{Elección de $M$}
Otra alternativa es disminuyendo el número de pasos de tiempo $M$. Cuando $U_0(t)$ es independiente del tiempo, el sistema debe llegar a un estado estacionario a partir del cual la solución deja de variar. Una vez en este estado, no tiene sentido continuar la ejecución. Entonces, se puede aplicar un criterio de convergencia que indique cuándo la solución se encuentra en el estado estacionario y la ejecución deba detenerse. El criterio tomado fue el siguiente: si la solución tiende a un estacionario independiente del tiempo, entonces se avanza la solución hasta que se cumplan al mismo tiempo las desigualdades
\[\frac{\left \|U^{k+1}-U^k \right \|_2}{\Delta t} < tol, \, \frac{\left \| V^{k+1} - V^k \right \|_2}{dt} < tol \]
con $tol$ tolerancia y $|| \bullet ||_2$ es la norma 2 matricial. Hay que tener en cuenta que a menor $tol$, mayor costo computacional debido a que se necesitan más pasos $M$. En este trabajo se decidió establecer $tol = 1 \times 10^{-5}$.
\ptitle{Algoritmo para $\Delta t$}

Además, si solo es de interés la solución en el estado estacionario, otra alternativa es eligir $\Delta t$ de modo que la solución converja más rápido. Esto último se basa en las hipótesis de que el estado estacionario no depende del paso de tiempo, lo cual debe ser estudiado en detalle, y que es posible llegar al mismo atravesando un estado transitorio lejos de la solución. Asumiendo ambas hipótesis, se propone el siguiente algoritmo para elegir $\Delta t$.
\begin{enumerate}
  \item Se parte de una propuesta $\Delta t_{guess}$ lo suficientemente pequeña para asegurar que la solución no diveja. En este trabajo se empleó $\Delta t_{guess} = 0.005.$ 
  \item Se avanza en 10 pasos la solución numérica, calculando en cada uno de ellos el parámetro $\epsilon$ definido como
  \[\epsilon = \sqrt{ \left \|U^{k+1}-U^k \right \|_2^2 +  \left \| V^{k+1} - V^k \right \|_2^2 }.\]
  Este parámetro da indicio de cuán rápido varían las velocidades.
  \item Se calcula la velocidad de convergencia $\chi_{guess}$ definida como la pendiente en un ajuste lineal de $\ln{\epsilon}$ vs número de paso de tiempo.
  \item Se divide el paso de tiempo por la mitad $\Delta t = \Delta t_{guess}/2$ y se calcula la velocidad de convergencia $\chi$ correspondiente. Si $\chi > \chi_{guess}$, se repite el proceso empleando $ \Delta t_{guess} = \Delta t$. Si $\chi < \chi_{guess}$, se define como mejor paso de tiempo a $\Delta t_{guess}$.
\end{enumerate}
Este algoritmo no es necesariamente el óptimo. Se basa en la hipótesis de que la velocidad de convergencia en los primeros pasos de tiempo es representativa de la velocidad en toda la evolución, lo cual no es cierto.


% \ptitle{Número de pasos $l_{simpler}$}

% Por último, a mayor $l_{simpler}$, mayor exactitud del método SIMPLER y mayor costo computacional. De este modo, suena lógico mantener $l_{simpler} = 1$. Sin embargo, como cambia la precisión del método, podría modificarse también la elección de $\Delta t$ y, consecuentemente, aumentar o disminuir el número de pasos de tiempo $M$ necesarios para llegar al estado estacionario. En definitiva, la relación entre $l_{simpler}$ y costo computacional podría no ser trivial y debe ser estudiada en detalle.


% \ptitle{Resumen de costo computacional}
% En este trabajo se estudiaron todos los métodos mencionados. En primer lugar, se determinó para $Re = 100$ y $1000$, el valor de $n_1$ tal que la solución numérica tiene un error menor al 5 \% respecto a una solución numérica de mayor precisión obtenida en \textcolor{red}{ref Guia}. En segundo lugar, estudiando la velocidad de convergencia en función de $\Delta t$ se planteó un algoritmo que permita elegir $\Delta t$ de modo de minimizar el número de pasos de tiempo necesarios. En tercer lugar, se decidió emplear el criterio de elección de $M$ durante todo el trabajo.

\ptitle{Resumen de lo que se va a estudiar}

En este trabajo se analizaron y emplearon las técnicas mencionadas anteriormente para reducir el tiempo de cómputo, para lo cual fue necesario en primer lugar verificar la hipótesis de la dependencia del estacionario con el paso de tiempo. Además, se calculó el mínimo valor de $n$ necesario para que la solución tenga un error menor al 5 \% respecto a la bibliografía \cite{Guia1982}. Por último, se estudió la dependencia del tiempo de cómputo con el número de iteraciones internas $l_{simpler}$.

Una vez finalizado el estudio sobre el costo computacional, se estudió la dependencia de la solución en el estado estacionario para distintos $Re$ respecto al método numérico del término advectivo, al tamaño de la discretización espacial $\Delta$ y al número de iteraciones internas $l_{simpler}$ en el algoritmo simpler. También se calculó el orden de convergencia de UP1 para distintos valores del número de Reynolds. Por último, se estudió el comportamiento de la solución en el tiempo respecto al método de evolución temporal. 

\section{Resultados y discusión}

\subsection{Costo computacional}

En primer lugar, se analizó la dependencia del estado estacionario con el paso de evolución temporal. El estado estacionario se define en base al criterio de convergencia descripto en la sección anterior con una tolerancia $tol = 1 \times 10^{-5}$. Se calculó $u(0.5,0.5)$ y $v(0.5,0.5)$ para $Re = 1000$, $n = 20$, $l_{simpler} = 1$ y distintos $\Delta t$ entre $0.005$ y $20$. Se emplearon diferencias centradas de orden 2 para el término advectivo y Euler implícito para la evolución temporal. En la figura \ref{fig:a_vel_vs_dt} se grafican $\delta u(0.5,0.5)$ y $\delta v(0.5,0.5)$ definidos como el valor absoluto de la diferencia entre la velocidad con determinado $\Delta t$ y su valor con $\Delta t = 0.005$. Se observa una diferencia entre los valores obtenidos del orden de $1 \times 10^{-4}$, ligeramente superior a la tolerancia del criterio de convergencia. Aunque si solo se consideraran $\Delta t < 1$ se obtendría una diferencia menor a dicha tolerancia y por lo tanto se podría decir que el estado estacionario no depende de $\Delta t$. De este modo, bajo estas condiciones es válido el uso del algortimo para calcular el mejor $\Delta t$ descripto en la sección anterior. Si bien la afirmación anterior sólo es válida en el caso particular considerado, se tomará como norma general en próximas simulaciones. En todos los casos analizados en este trabajo se empleó el algoritmo para buscar $\Delta t$, salvo que se aclare lo contrario.

\begin{figure}[h]
  \includegraphics[clip=true,width=\columnwidth]{a_vel_vs_dt.pdf}
  \caption{$\delta u(0.5,0.5)$ y $\delta v(0.5,0.5)$ en función del paso temporal $\Delta t$. Se empleó $Re = 1000$, $n = 20$ y $l_{simpler} = 1$. Además, se utilizó diferencias finitas centradas de orden 2 para el término advectivo y difusivo y Euler implícito para la evolución temporal.}
   \label{fig:a_vel_vs_dt}
\end{figure}

\ptitle{Resumen}
Otro modo de disminuir el costo computacional es reduciendo el número de variables involucradas. Se empleó el algoritmo de reducción de $n$ presentado en la sección anterior para $Re = 100$ y $1000$. Se empleó $l_{simpler} = 1$ y EI para la evolución temporal. Mientras que para el término advectivo se empleó UP1 y DC2 para el advectivo. En base a esto, se empleó $n_{guess} = 70$ para UP1 y $n_{guess} = 32$ para DC2. Esto último considerando que los órdenes de convergencia de ambos métodos son distintos y se esperaría que para UP1 se necesite mayor $n$ (y consecuentemente menor $\Delta$) para obtener la misma precisión.

En cada ejecución se midió el tiempo de cómputo $T$ para encontrar $u(0.5,0.5)$ en el estado estacionario con un error menor al 5 \%. Los resultados se resumen en la tabla \ref{tabla:costo_computacional}. En primer lugar, el método UP1 es consistentemente más caro que el DC2. Esto se podría deber a que UP1 es de un orden de convergencia menor y por lo tanto exige $n$ mayor. En segundo lugar, el costo aumenta con $Re$. Esto podría deberse a que ambos métodos presentan mayor error al aumentar Re. Por último, es necesario aclarar que el tiempo de cómputo en el caso UP1 y $Re = 100$ se debió en su mayoría a la minimización de $n$. Esto se debe a que el algoritmo requiere evaluar el estado estacionario para cada $n$ desde $n_guess$ hasta $n = 56$ en este caso. Además, en el caso UP1 y $Re = 1000$ no se logró un error menor al 5 \% en un tiempo de ejecución menor a la hora.


\begin{table}[]
  \begin{tabular}{|c|c|c|c|c|c|}
  \hline
  \begin{tabular}[c]{@{}c@{}}método para el\\ termino advectivo\end{tabular} & $Re$ & $n$ & $\Delta t$ & $N$ & $T [min]$ \\ \hline
  DC2 & 100 & 20 & 2.56 & 22 & 4.5 \\ \hline
  UP1 & 100 & 56 & 0.64 & 47 & 45 \\ \hline
  DC2 & 1000 & 30 & 0.16 & 768 & 11.3 \\ \hline
  UP1 & 1000 & $\geqslant  70$ & 0.08 & 1024 & $\geqslant  48$ \\ \hline
  \end{tabular}
  \caption{Tiempo de ejecución $T$ en minutos para avanzar la solución hasta el estado estacionario bajo distintas condiciones. $Re$ es el número de Reynolds, $n$ es el número de volúmenes en cada dirección espacial, $\Delta t$ es el paso de tiempo, $N$ el número de pasos de tiempo ejecutados. El tiempo se midió con un procesador AMD Ryzen 7 4700U y 8 Gb de memoria RAM. El caso UP1 con $Re = 1000$ presentó un error de aproximadamente 70 \%.}
  \label{tabla:costo_computacional}
\end{table}

% \ptitle{Para qué se usó el algoritmo anterior}
% \begin{itemize}
%   \item Este algoritmo no siempre va a encontrar el mejor $\Delta t$ posible, pero se espera que esté cerca del óptimo.
%   \item El algoritmo anterior fue empleado para hallar el estado estacionario en todo este trabajo salvo el caso particular $n_1 = 80$. Este es el caso más costoso y se buscará a mano el mejor $\Delta t$ para cada $Re$ específico.
%   \item Se usaron $\Delta t = 0.35$, $1.25$ y $2.0$ para $Re = 100$, $1000$ y $5000$, respecticamente
% \end{itemize}

Se estudió la dependencia del paso de tiempo y el tiempo de cómputo en función del número de iteraciones internas del algoritmo $l_{simpler}$ para avanzar la solución hasta el estado estacionario. Se emplearon las mismas condiciones que la simulación anterior a excepción del número de volúmenes finitos fijo en $n = 30$. Los resultados se resumen en la tabla \ref{tabla:dt_costo_vs_lsimpler}. Se observa que el número de pasos de tiempo necesarios para alcanzar el estado estacionario varía ligeramente con $l_{simpler}$, no así el tiempo de cómputo que aumenta de forma proporcional con el mismo. Por otro lado, el valor de $\Delta t$ calculado mediante el algoritmo es independiente de $l_{simpler}$.


% \begin{figure}[h]
%   \includegraphics[clip=true,width=\columnwidth]{dt_costo_vs_lsimpler.pdf}
%   \caption{}
%    \label{fig:dt_costo_vs_lsimpler}
% \end{figure}



\begin{table}[]
  \begin{tabular}{|c|c|c|c|}
  \hline
  \multicolumn{1}{|c|}{$l_{simpler}$ \quad} & \multicolumn{1}{c|}{$N$\quad} & \multicolumn{1}{c|}{$\Delta t$\quad} & \multicolumn{1}{c|}{$T [min]$} \\ \hline
  1 & 756 & 0.16 & 5.8 \\ \hline
  2 & 793 & 0.16 & 12.7 \\ \hline
  3 & 800 & 0.16 & 18.8 \\ \hline
  \end{tabular}
  \caption{Tiempo de ejecución $T$ en minutos para alcanzar el estado estacionario bajo distinto número de iteraciones internas $l_{simpler}$. $N$ es el número de pasos de tiempo ejecutados y $\Delta$, el paso de tiempo.}
  \label{tabla:dt_costo_vs_lsimpler}
\end{table}


\subsection{Esquema numérico}

Se estudió la solución numérica respecto al número de Reynolds $Re$, al tamaño de la discretización espacial $\Delta$, al método numérico empleado para el término advetivo, al número de iteraciones internas $l_{simpler}$ y al método de evolución temporal. Además, se calculó el orden de convergencia del esquema UP1 para el término advectivo. En todos los casos se empleó el algoritmo descripto en la sección anterior para encontrar el mejor paso de tiempo $\Delta t$, salvo que sea indicado lo contrario, y el método DC2 para el término difusivo.

% En primer lugar, se estudió en detalle el comportamiento de la solución con el método numérico empleado para el término advectivo. Se implementaron los esquemas DC2, UP1 y QUICK y se calculó el estado estacionario para $Re = 100, 1000 \, \mathrm{y} \, 5000$, para $n1 = 20, 40 \, \mathrm{y} \, 80$ y para los tres esquemas mencionados. Se empleó DC2 para el término difusivo y Euler implícito para la evolución temporal.

\subsubsection{Dependencia con el tamaño de la discretización $\Delta$}

\ptitle{Comparación DC2 vs n1 para distintos Re}
Se estudió la dependencia del estado estacionario con la discretización espacial $\Delta$ y el número de $Re$, empleando DC2 para el término advectivo, $l_{simpler} = 1$ y Euler implícito para la evolución temporal. Para todos los casos se determinó $\Delta t$ con el algoritmo mencionado, salvo para $\Delta = 0.0125$. Este es el caso más caro computacionalmente, tanto para el uso del algoritmo como para la búsqueda del estado estacionario. Se buscó el mejor paso de tiempo a mano, encontrando $\Delta t = 0.35$, $1.25$ y $2.0$ para $Re = 100$, $1000$ y $5000$, respectivamente. En las figuras \ref{fig:velocidades_u_DC2_vs_Re} y \ref{fig:velocidades_v_DC2_vs_Re} se grafican las velocidades $u(0.5,y)$ y $v(x,0.5)$ para los distintos casos. Para $Re = 100$ se observa que el resultado varía cualitativamente poco con el tamaño de la discretización. Para $Re = 1000$ la diferencia entre los resultados es significativa, llegando en algunos sitios a superar la tolerancia del criterio de convergencia. Por último, para $Re = 5000$ todas las soluciones distan de la referencia. Esto indica que a mayor número de Reynolds es necesario disminuir el tamaño de la discretización, es decir, aumentar $n$, para obtener una solución más precisa.

\subsubsection{Dependencia con el método numérico para el término advectivo \label{sec:termino_adv}}

\ptitle{Comparación cualitativa entre métodos numéricos para distintos Re}
Se estudió la dependencia del estado estacionario con el número de Reynolds y el método numérico para el término advectivo. En todos los casos se empleó $n = 80$ y los valores de $\Delta t$ específicos presentados en anteriormente. En las figuras \ref{fig:terminos_advs_u} y \ref{fig:terminos_advs_u} se grafican las velocidades $u(0.5,y)$ y $v(x,0.5)$ para los distintos casos. Cualitativamente, para $Re = 100$ no se observan grandes diferencias. Para $Re = 1000$ es notable que el método UP1 no aproxima correctamente la solución, mientras que DC2 y QUICK sí lo hacen. Esto se puede justificar en los órdenes de aproximación de los métodos, siendo el primero de orden 1 y los segundos de orden \textcolor{red}{2}. Esto también justifica por qué en la tabla \ref{tabla:costo_computacional} fue necesario un valor de $n$ mayor para UP1 que para DC2. Para $Re = 5000$ ninguno de los métodos aproxima correctamente la solución en la totalidad del dominio. En base a los resultados obtenidos al variar $n$, esto podría deberse al tamaño de la grilla espacial y no al método numérico.

Cabe preguntarse por qué a medida que aumenta $Re$ la solución con UP1 se aleja aún más de la de referencia. A mayor $Re$, el término advectivo en las ecuaciones de momento \ref{eq:momento_x} y \ref{eq:momento_y} gana imporancia respecto al difusivo. De este modo, se esperaría una solución con baja difusión. Sin embargo, el método UP1 presenta naturalmente difusión, como se mencionó en la sección anterior, y de este modo la solución numérica se aparta de la de referencia.

\ptitle{Comparación numérica entre métodos numéricos para distintos $Re$.}
El análisis anterior es meramente cualitativo. A continuación se realiza un estudio cuantitativo para determinar cuál es el mejor método numérico. Para cada variación posible de $n$, $Re$ y método para el término advectivo se calculó el error relativo $e_{rel}$ respecto a la referencia \cite{Guia1982} en el centro del dominio $x = 0.5$ e $y = 0.5$. Para calcular este error se suman cuadráticamente los errores en $u$ y $v$ y se normaliza con el valor de referencia, es decir,
\[e_{rel} = \frac{\sqrt{(u_{exp} - u_{sol})^2 + (v_{exp} - v_{sol})^2}}{\sqrt{u_{sol}^2 + v_{sol}^2}}. \]
donde todos las velocidades están evaluadas en el centro. Para calcular las velocidades en el centro se promediaron los valores en la cercanía, resultando en una aproximación de segundo orden. Los resultados se resumen en la figura \ref{fig:termino_advectivo}. Al comportamiento visto en las figuras \ref{fig:terminos_advs_u} y \ref{fig:terminos_advs_u} se agregan dos particularidades. En primer lugar, independientemente de $Re$ el método UP1 presenta mayor error que los demás, lo cual es esperable por ser un método de menor orden. En segundo lugar, en casi todos los casos a mayor $n$, menor error. La excepción son los casos de $Re = 100$, $n = 80$ y DC2 y QUICK como método para el término advectivo.

Por último, cabe preguntarse qué método es más preciso. Para tal fin se resumen en la tabla \ref{tabla:erel} los errores de cada método para $n = 80$ y $Re = 100$, $1000$ y $5000$. Se observa que el método DC2 posee un error ligeramente menor a QUICK. Entonces, se podría argumentar que DC2 es un método más preciso que QUICK. Sin embargo, esta afirmación debe tomarse con cuidado debido a que sólo se analizó la velocidad en el punto central y no en todo el dominio. En un análisis formal podría obtenerse el comportamiento opuesto.

% Mencionar cómo se calcula $u(0.5,0.5)$ y que sólo tiene sentido calcularlo allí porque la aproximación es del mismo orden que los métodos. Calcularla en otro lado con un promedio ponderado es de mayor orden.




% \onecolumngrid


\begin{figure}
  \centering
  \begin{subfigure}[b]{0.32\textwidth}
      \centering
      \includegraphics[width=\textwidth]{termino_adv_Db_u_100.pdf}
      \caption{}
      \label{fig:termino_adv_Db_u_100}
  \end{subfigure}
  \hfill
  \begin{subfigure}[b]{0.32\textwidth}
      \centering
      \includegraphics[width=\textwidth]{termino_adv_Db_u_1000.pdf}
      \caption{}
      \label{fig:termino_adv_Db_u_1000}
  \end{subfigure}
  \hfill
  \begin{subfigure}[b]{0.32\textwidth}
      \centering
      \includegraphics[width=\textwidth]{termino_adv_Db_u_5000.pdf}
      \caption{}
      \label{fig:termino_adv_Db_u_5000}
  \end{subfigure}
     \caption{Velocidad $u(0.5,y)$ en el estado estacionario en función de $y$ para distinto tamaño de la discretización espacial $\Delta$ y número de Reynolds $Re$: (a) $Re = 100$, (b) $Re = 1000$ y (c) $Re = 5000$. Nótese que en \ref{fig:termino_adv_Db_u_100} las tres curvas se encuentran superpuestas.}
     \label{fig:velocidades_u_DC2_vs_Re}
\end{figure}

\begin{figure}
  \centering
  
  \begin{subfigure}[b]{0.32\textwidth}
    \centering
    \includegraphics[width=\textwidth]{termino_adv_Db_v_100.pdf}
    \caption{}
    \label{fig:termino_adv_Db_v_100}
\end{subfigure}
\hfill
\begin{subfigure}[b]{0.32\textwidth}
    \centering
    \includegraphics[width=\textwidth]{termino_adv_Db_v_1000.pdf}
    \caption{}
    \label{fig:termino_adv_Db_v_1000}
\end{subfigure}
\hfill
\begin{subfigure}[b]{0.32\textwidth}
    \centering
    \includegraphics[width=\textwidth]{termino_adv_Db_v_5000.pdf}
    \caption{}
    \label{fig:termino_adv_Db_v_5000}
\end{subfigure}
     \caption{Velocidad $v(x,0.5)$ en el estado estacionario en función de $x$ para distinto tamaño de la discretización espacial $\Delta$ y número de Reynolds $Re$: (a) $Re = 100$, (b) $Re = 1000$ y (c) $Re = 5000$. Nótese que en \ref{fig:termino_adv_Db_v_100} las curvas correspondientes a $\Delta = 0.0125$ y $0.025$ se encuentran superpuestas.}
     \label{fig:velocidades_v_DC2_vs_Re}
\end{figure}






\begin{figure}
  \centering
  \begin{subfigure}[b]{0.32\textwidth}
      \centering
      \includegraphics[width=\textwidth]{terminos_advs_u_100.pdf}
      \caption{}
      \label{fig:terminos_advs_u_100}
  \end{subfigure}
  \hfill
  \begin{subfigure}[b]{0.32\textwidth}
      \centering
      \includegraphics[width=\textwidth]{terminos_advs_u_1000.pdf}
      \caption{}
      \label{fig:terminos_advs_u_1000}
  \end{subfigure}
  \hfill
  \begin{subfigure}[b]{0.32\textwidth}
      \centering
      \includegraphics[width=\textwidth]{terminos_advs_u_5000.pdf}
      \caption{}
      \label{fig:terminos_advs_u_5000}
  \end{subfigure}
     \caption{Velocidades $u(0.5,y)$ en el estado estacionario en función de $y$ para distintos métodos numéricos para el término advectivo y distintos números de Reynolds $Re$: (a) $Re = 100$, (b) $Re = 1000$ y (c) $Re = 5000$. En todos los casos se empleó una grilla espacial de tamaño $n \times n$ con $n = 80$ y el método Euler implícito para la evolución temporal. Nótese que las curvas correspondientes a los métodos QUICK y DC2 se encuentran superpuestas en (a) y(b).}
     \label{fig:terminos_advs_u}
\end{figure}

\begin{figure}
  \centering
  \begin{subfigure}[b]{0.32\textwidth}
    \centering
    \includegraphics[width=\textwidth]{terminos_advs_v_100.pdf}
    \caption{}
    \label{fig:terminos_advs_v_100}
  \end{subfigure}
  \hfill
  \begin{subfigure}[b]{0.32\textwidth}
    \centering
    \includegraphics[width=\textwidth]{terminos_advs_v_1000.pdf}
    \caption{}
    \label{fig:terminos_advs_v_1000}
  \end{subfigure}
  \hfill
  \begin{subfigure}[b]{0.32\textwidth}
    \centering
    \includegraphics[width=\textwidth]{terminos_advs_v_5000.pdf}
    \caption{}
    \label{fig:terminos_advs_v_5000}
  \end{subfigure}
     \caption{Velocidad $v(x,0.5)$ en el estado estacionario en función de $x$ para distintos métodos numéricos para el término advectivo y distintos números de Reynolds $Re$: (a) $Re = 100$, (b) $Re = 1000$ y (c) $Re = 5000$. En todos los casos se empleó una grilla espacial de tamaño $n \times n$ con $n = 80$ y el método Euler implícito para la evolución temporal. Nótese que las curvas correspondientes los métodos QUICK y DC2 se encuentran superpuestas en (a) y (b).}
     \label{fig:terminos_advs_v}
\end{figure}



\begin{figure}
  \centering
  \begin{subfigure}[b]{0.32\textwidth}
      \centering
      \includegraphics[width=\textwidth]{termino_adv_DC2.pdf}
      \caption{}
      \label{fig:termino_adv_DC2}
  \end{subfigure}
  \hfill
  \begin{subfigure}[b]{0.32\textwidth}
      \centering
      \includegraphics[width=\textwidth]{termino_adv_UP1.pdf}
      \caption{}
      \label{fig:termino_adv_UP1}
  \end{subfigure}
  \hfill
  \begin{subfigure}[b]{0.32\textwidth}
      \centering
      \includegraphics[width=\textwidth]{termino_adv_QUICK.pdf}
      \caption{}
      \label{fig:termino_adv_QUICK}
  \end{subfigure}
     \caption{Error relativo $e_{rel}$ para distintos número de Reynolds $Re$, tamaño de la discretización espacial $\Delta$ y método numérico para el término advectivo: (a) Diferencias centradas de orden 2, (b) Upwind de orden 1 y (c) esquema QUICK. En todos los casos se empleó Euler implícito para la evolución temporal.}
     \label{fig:termino_advectivo}
\end{figure}



% \twocolumngrid



\begin{table}[]
  \begin{tabular}{l|lll|}
  \cline{2-4}
            & \multicolumn{3}{c|}{Re}                                     \\ \cline{2-4} 

            & \multicolumn{1}{c|}{100} & \multicolumn{1}{c|}{1000} &  \multicolumn{1}{c|}{5000} \\ \hline
  \multicolumn{1}{|l|}{DC2} & \multicolumn{1}{c|}{$0.08541$} & \multicolumn{1}{c|}{$0.17567$} & $0.69045$ \\ \hline
  \multicolumn{1}{|l|}{UP1} & \multicolumn{1}{c|}{$0.14813$} & \multicolumn{1}{c|}{$3.46833$} & $17.6402$ \\ \hline
  \multicolumn{1}{|l|}{QUICK} & \multicolumn{1}{c|}{$0.08778$} & \multicolumn{1}{c|}{$0.30266$} & $4.09139$ \\ \hline
  \end{tabular}
  \caption{Error relativo $e_{rel}$ para distintos número de Reynolds $Re$ y método numérico para el término advectivo: (a) Diferencias centradas de orden 2, (b) Upwind de orden 1 y (c) método QUICK. En todos los casos se empleó el método Euler implícito para la evolución temporal.}
  \label{tabla:erel}
\end{table}

\subsubsection{Órden de convergencia del método UP1}

Se calculó el orden de convergencia espacial del método UP1 para el término advectivo para $Re = 1$ y $1000$. Para ello se calculó el error en la determinación de $u(0.5, 0.2)$ y $v(0.2, 0.5)$ respecto a una solución precisa para distintos valores de $\Delta$. Se consideró como solución precisa a aquella con $n = 80$ y DC2 para el término advectivo, considerado como el método más preciso de los tres en base a los resultados presentados anteriormente. Se empleó el método Euler implícito para la evolución temporal.

Los resultados se grafican en la figura \ref{fig:error_UP1_n1_vs_Re}. Comparando \ref{fig:error_UP1_n1_vs_Re_1} con \ref{fig:error_UP1_n1_vs_Re_1000} se observa que el método presenta mayor error con $Re = 1000$ que con $Re = 1$, consistente con lo estudiado en \ref{sec:termino_adv}. En base al comportamiento lineal para ambas velocidades se realizó un ajuste de los datos, obteniendo como pendiente el orden de convergencia para $u$ y $v$ y $Re = 1$ y $1000$. En el caso de $Re = 1$ se obtuvo un orden de convergencia para $u$ de $2.5 \pm 0.3$ y para $v$ de $2.9 \pm 0.4$. Los órdenes de convergencia para cada variable son indistinguibles entre sí y mayores al orden teórico de UP1. Esto podría deberse a que a bajo número de Reynolds el término difusivo en las ecuaciones \ref{eq:momento_x} y \ref{eq:momento_y} cobra mayor importancia que el advectivo. Como el difusivo se discretizó con DC2 se esperaría un error del orden 2. Sin embargo, el orden de convergencia es aún mayor. No es claro a qué se debe este comportamiento pero podría estar relacionado a la mezcla de métodos espaciales y temporales utilzada y además, a que la solución de referencia no es la solución exacta sino una aproximación. En el caso de $Re = 1000$ se obtuvo un orden de convergencia para $u$ de $0.52 \pm 0.06$ y para $v$ de $0.52 \pm 0.05$. Ambos valores son indistinguibles entre sí y menores al orden teórico UP1. Análogamente al caso anterior, como aumentó el número de Reynolds, el término advectivo cobra mayor importancia y se esperaría un orden 1 de convergencia. Sin embargo, el orden es aún menor. No es claro a qué se debe este comportamiento pero podría deberse a las mismas razones discutidas para $Re = 1$.

\begin{figure}
  \centering
  \begin{subfigure}[b]{0.45\textwidth}
      \centering
      \includegraphics[width=\textwidth]{error_UP1_n1_vs_Re_1.pdf}
      \caption{}
      \label{fig:error_UP1_n1_vs_Re_1}
  \end{subfigure}
  \hfill
  \begin{subfigure}[b]{0.45\textwidth}
      \centering
      \includegraphics[width=\textwidth]{error_UP1_n1_vs_Re_1000.pdf}
      \caption{}
      \label{fig:error_UP1_n1_vs_Re_1000}
  \end{subfigure}
     \caption{Error local de $u$ y $v$ calculados con Upwind de orden 1 respecto a la solución con diferencias centradas de orden 2 para distintos valores de $\Delta$. Se analizaron los casos (a) $Re = 1$ y (b) $Re = 1000$. Se empleó el método Euler implícito para la evolución temporal y diferencias centradas de orden 2 para el término de difusión. Se ajustó en todos los casos una dependencia lineal $a \mathrm{log(\Delta)} + b$, donde $a$ indica el orden de convergencia. En el caso $Re = 1$ se obtuvo para $u$ una pendiente $a_{u}^1 = 2.5 \pm 0.3$ y para $v$, $a_{v}^1 = 2.9 \pm 0.4$. Mientras que en el caso $Re = 1000$ se obtuvo para $u$ una pendiente $a_u^{1000} = 0.52 \pm 0.06$ y para $v$, $a_v^{1000} = 0.52 \pm 0.05$.
     }
     \label{fig:error_UP1_n1_vs_Re}
\end{figure}

\subsubsection{Dependencia con el número de iteraciones del algoritmo SIMPLER.}

Cabe preguntarse cómo cambia la precisión con $l_{simpler}$. En la figura \ref{fig:lsimpler} se grafican las velocidades $u(0.5,y)$ y $v(x,0.5)$ en el estado estacionario para los distintos valores de $l_{simpler}$. Cualitativamente las velocidades en el estado estacionario no se modifican con $l_{simpler}$. Sin embargo, la solución numérica dista de la de referencia. Esto podría deberse al valor de $n$ utilizado o a los métodos numéricos en sí.

\begin{figure}
  \centering
  \begin{subfigure}[b]{0.45\textwidth}
      \centering
      \includegraphics[width=\textwidth]{f_lsimpler_u.pdf}
      \caption{}
      \label{fig:lsimpler_u}
  \end{subfigure}
  \hfill
  \begin{subfigure}[b]{0.45\textwidth}
      \centering
      \includegraphics[width=\textwidth]{f_lsimpler_v.pdf}
      \caption{}
      \label{fig:lsimpler_v}
  \end{subfigure}
     \caption{Velocidades $u(0.5,y)$ y $v(x,0.5)$ en el estado estacionario para $l_simpler = 1$, $2$ y $3$ iteraciones internas en el algoritmo SIMPLER. En ambos casos todas las curvas están superpuestas. Se empleó $Re = 1000$ y $n = 30$. Además, se utilizó diferencias finitas centradas de orden 2 para el término advectivo y difusivo y Euler implícito para la evolución temporal. Nótese que en ambas figuras las tres curvas se encuentran superpuestas.}
     \label{fig:lsimpler}
\end{figure}

\subsubsection{Dependencia con el método numérico de evolución temporal}
Se determinó el efecto del método numérico de evolución temporal sobre la solución. Para esto se modificó la condición de contorno en la cara superior para que sea dependiente del tiempo como $U_0 = cos(t)$. Se calculó la evolución temporal de $u(0.5,0.5)$ y $v(0.5,0.5)$ hasta $t = 80$ empleando los métodos Euler implícito y Crack-Nicholson. En cuanto a los parámetros de la simulación se empleó $n = 30$ y $Re = 1000$, DC2 para los término advectivo y difusivo, $\Delta t = 0.4$ y $l_{simpler} = 1$ y $3$. Los resultados se resumen en la figura \ref{fig:e}.
  
En el caso de la velocidad $u(0.5,0.5)$, se observa que la solución con EI y $l_{simpler} = 1$ converge rápidamente, aunque difiere de las demás con mayor precisión. Estas se encuentran prácticamente superpuestas con algunas diferencias en los mínimos y máximos y hasta $t \approx 40$ no están convergidas. Es notable que cualitativamente una solución con EI y $l_{simpler} = 3$ tiene prácticamente el mismo comportamiento que una con CN, a pesar que este último método es de mayor orden.
  
En el caso de la velocidad $v(0.5,0.5)$, la solución con EI y $l_{simpler} = 1$ también dista de las demás. A diferencia de la velocidad horizontal, la solución con CN y $l_{simpler} = 1$ difiere de las demás con mayor precisión. La diferencia fundamental se encuentra en el número de iteraciones internas $l_{simpler}$. En resumen, cualitativamente se observa que la precisión está fuertemente influenciada por el número de iteraciones internas $l_{simpler}$.


\begin{figure}
  \centering
  \begin{subfigure}[b]{0.45\textwidth}
      \centering
      \includegraphics[width=\textwidth]{e_1.pdf}
      \caption{}
      \label{fig:e_1}
  \end{subfigure}
  \hfill
  \begin{subfigure}[b]{0.45\textwidth}
      \centering
      \includegraphics[width=\textwidth]{e_2.pdf}
      \caption{}
      \label{fig:e_2}
  \end{subfigure}
     \caption{Evolución de (a) $u(0.5,0.5)$ y (b) $v(0.5,0.5)$ en el tiempo. Se empleó como condición de contorno $U_0 = cos(t)$ y $Re = 1000$. En cuanto a las condiciones de la simulación se utilizó $n = 30$, $\Delta t = 0.4$, diferencias centradas de orden 2 para los términos advectivo y difusivo, distintos métodos para la evolución temporal ('EI': Euler Implícito, 'CN': Crank-Nicholson) y distinto número de iteraciones internas $l_{simpler}$ del algoritmo SIMPLER. Nótese que en (a) las curvas correspondientes a EI y $l_{simpler}$ = 1, CN y $l_{simpler}$ = 1 y 3, se encuentran superpuestas. Además, en (b) las curvas correspondientes a CN y EI con $l_{simpler} = 3$ se encuentran superpuestas.}
     \label{fig:e}
\end{figure}









\section{Conclusión}


Se resolvió el problema de la cavidad cuadrada hidrodinámica bidimensional mediante el método de volumenes finitos y algoritmo SIMPLER calculando presión y velocidades con grilla desplazada. Se estudiaron distintos métodos numéricos espaciales y de evolución temporal. En particular, se analizaron los esquemas de diferencias finitas de orden 2 (DC2), Upwind de orden 1 (UP1) y el esquema QUICK  para el término advectivo de la ecuación de momento. Mientras que para la evolución temporal, los métodos Euler implícito y Crank-Nicholson. 

Debido a la cantidad de variables involucradas, la resolución conlleva un gran costo computacional. Se plantearon algoritmos que permitiesen minimizar el tiempo de cómputo. En primer lugar, bajo la hipótesis de que el estado estacionario no depende del paso de tiempo, se propuso un algoritmo para elegir el paso de tiempo de modo de maximizar la velocidad de convergencia durante los primeros 10 pasos. Para determinar cuándo el sistema se encuentra en el estado estacionario se definió un criterio de convergencia empleando las derivadas de las velocidades. En segundo lugar, se propuso un algoritmo para disminuir progresivamente el tamaño de la grilla espacial hasta obtener un error menor al 5 \% respecto a la bibliografía. Se obtuvo que UP1 es consistentemente más caro computacionalmente que DC2 y, además, el costo aumenta con el número de Reynolds. En tercer lugar, se analizó la dependencia del tiempo de cómputo con el número de iteraciones internas en el algoritmo simpler. Se obtuvo que el tiempo de cómputo aumenta de forma aproximadamente proporcional al número de iteraciones internas. Sin embargo, cualitativamente el estado estacionario no presenta diferencias significativas al modificar tal número.


Por otro lado, se analizó la dependencia del estado estacionario con el número de Reynolds y el método numérico para el término advetivo, empleando como condición de contorno una velocidad constante en el tiempo. Se obtuvo que la precisión disminuye con el número de Reynolds independientemente del método para el término advectivo. Además, que el método con mayor error es UP1 debido a que es de menor orden que los demás y presenta difusión, cada vez más notoria cuando aumenta $Re$.

Además, se determinó el método más exacto para el término advectivo. Para esto se calculó el error relativo en la determinación de $u(0.5,0.5)$ y $v(0.5,0.5)$ respecto a la bibliografía. Se obtuvo que el método con menor error es DC2. Luego, se lo empleó como solución de referencia para calcular el órden de convergencia del método UP1 en la determinación de $u(0.5,0.2)$ y $v(0.2,0.5)$. Se obtuvo para $Re = 1$ un orden de convergencia mayor a 2 y para $Re = 1000$ un orden menor a 1. El primero se justifica en que a bajo número de Reynolds el término advectivo discretizado con UP1 tiene menor importancia que el difusivo discretizado con DC2. Mientras que el segundo se justifica de forma contraria: a mayor Reydolds el término advectivo cobra importancia respecto al difusivo.

Por último, se cambió la condición de contorno a una oscilatoria dependiente del tiempo y se resolvió el problema con los métodos Euler implícito y Crank-Nicholson, empleando $l_{simpler} = 1$ y $3$ iteraciones internas en el algoritmo SIMPLER. Cualitativamente se observó que la precisión está influenciada tanto por el método numérico como por el número de iteraciones del algoritmo SIMPLER.





\bibliography{Chehade_final.bib}

\end{document}





