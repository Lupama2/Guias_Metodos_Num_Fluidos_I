\documentclass[aps,prb,twocolumn,superscriptaddress,floatfix,longbibliography]{revtex4-2}

\usepackage[utf8]{inputenc}
\usepackage[spanish]{babel}
\usepackage{graphicx}
\usepackage{amsmath}
\usepackage{subcaption}
\usepackage{wrapfig} 
\usepackage[export]{adjustbox}

\usepackage{amsmath,amssymb} % math symbols
\usepackage{bm} % bold math font
\usepackage{graphicx} % for figures
\usepackage{comment} % allows block comments
\usepackage{textcomp} % This package is just to give the text quote '
%\usepackage{ulem} % allows strikeout text, e.g. \sout{text}

\usepackage[spanish]{babel}

\usepackage{enumitem}
\setlist{noitemsep,leftmargin=*,topsep=0pt,parsep=0pt}

\usepackage{xcolor} % \textcolor{red}{text} will be red for notes
\definecolor{lightgray}{gray}{0.6}
\definecolor{medgray}{gray}{0.4}

\usepackage{hyperref}
\hypersetup{
colorlinks=true,
urlcolor= blue,
citecolor=blue,
linkcolor= blue,
bookmarks=true,
bookmarksopen=false,
}

% Code to add paragraph numbers and titles
\newif\ifptitle
\newif\ifpnumber
\newcounter{para}
\newcommand\ptitle[1]{\par\refstepcounter{para}
{\ifpnumber{\noindent\textcolor{lightgray}{\textbf{\thepara}}\indent}\fi}
{\ifptitle{\textbf{[{#1}]}}\fi}}
%\ptitletrue  % comment this line to hide paragraph titles
%\pnumbertrue  % comment this line to hide paragraph numbers

% minimum font size for figures
\newcommand{\minfont}{6}

% Uncomment this line if you prefer your vectors to appear as bold letters.
% By default they will appear with arrows over them.
% \renewcommand{\vec}[1]{\bm{#1}}

%Cambiar Cuadros por Tablas y lista de...
%\renewcommand{\listtablename}{Índice de tablas}
\renewcommand{\tablename}{Tabla}
\renewcommand{\date}{Fecha}

%\graphicspath{ {C:/Users/lupam/Mi unidad/Pablo Chehade/Instituto Balseiro (IB)/Métodos numéricos en fluidos I/Metodos_Num_Fluidos_I/Guias/Guia_1/Informe/Figures} } %Para importar imágenes desde una carpeta


\usepackage[bottom]{footmisc} %para que las notas al pie aparezcan en la misma página



\begin{comment}

%Comandos de interés:

* Para ordenar el documento:
\section{Introducción}
\section{\label{sec:Formatting}Formatting} %label para luego hacer referencia a esa sección

\ptitle{Start writing while you experiment} %pone nombre y título al documento dependiendo de si en el header están los comandos \ptitletrue y \pnumbertrue

* Ecuaciones:
\begin{equation}
a^2+b^2=c^2 \,.
\label{eqn:Pythagoras}
\end{equation}

* Conjunto de ecuaciones:
\begin{eqnarray}
\label{eqn:diagonal}
\nonumber d & = & \sqrt{a^2 + b^2 + c^2} \\
& = & \sqrt{3^2+4^2+12^2} = 13
\end{eqnarray}

* Para hacer items / enumerar:
\begin{enumerate}
  \item
\end{enumerate}

\begin{itemize}
  \item
\end{itemize}

* Figuras:
\begin{figure}[h]
    \includegraphics[clip=true,width=\columnwidth]{pixel-compare}
    \caption{}
     \label{fig:pixels}
\end{figure}

* Conjunto de figuras:
(no recuerdo)


* Para hacer referencias a fórmulas, tablas, secciones, ... dentro del documento:
\ref{tab:spacing}

* Para citar
Elementos de .bib
\cite{WhitesidesAdvMat2004}
url
\url{http://www.mendeley.com/}\\

* Agradecimientos:
\begin{acknowledgments}
We acknowledge advice from Jessie Zhang and Harry Pirie to produce Fig.\ \ref{fig:pixels}.
\end{acknowledgments}

* Apéndice:
\appendix
\section{\label{app:Mendeley}Mendeley}

* Bibliografía:
\bibliography{Hoffman-example-paper}

\end{comment}



\begin{document}

% Allows to rewrite the same title in the supplement
\newcommand{\mytitle}{Resumen de la materia}

\title{\mytitle}

\author{Pablo Chehade \\
    \small \textit{pablo.chehade@ib.edu.ar} \\
    \small \textit{Métodos Numéricos en Fluidos I, Instituto Balseiro, CNEA-UNCuyo, Bariloche, Argentina, 2022} \\}


\begin{abstract}
El objetivo es hacer un cuadro sinóptico de la materia como la hoja de fórmulas
\end{abstract}

\maketitle


\section{PVI}
\begin{itemize}
    \item Una ec. de orden n se puede transformar en n ecuaciones de orden 1
\end{itemize}

\subsubsection{Problema en $R^1$}
\begin{itemize}
    \item Problema gral: encontrar $y(t): R \rightarrow R /$
    $\frac{dy}{dt} = f(y,t); t>0$
    $y(0) = y_0$
\end{itemize}

\subsubsubsection{Error}
Hay que diferenciar entre
\begin{itemize}
    \item Error local: error en el paso $n$ asumiento que $y_{n-1}$ es conocido. Se obtiene a partir de serie de Taylor sobre el esquema numérico y comparando con la serie de Taylor correcta.
    \item Error global: error en un punto considerando el error de aproximación en todos los $y_n$ anteriores. Si un método es de orden local $p$, entonces el error global será de orden o, de otra manera, ''converge con velocidad'', $p-1$.
\end{itemize}

\subsubsubsection{Estabilidad numérica}
En la práctica se dice que un método es
\begin{itemize}
    \item Inestable
    \item Estable: si no magnifica los errores que aparecen. Puede explotar la solución como no.
    \item Condicionalmente estable: condición sobre $\Delta t$
\end{itemize}

\subsubsubsubsection{Estabilidad lineal}
Es un modo de estudiar la estabilidad de un método. Propone estudiar la ecuación modelo $y' = \lambda y$ cuya solución es $y(t) = y_0 e^{\lambda t}$. Permite estudiar para soluciones reales acotadas si la solución numérica es acotada

\textcolor{red}{Tengo que ver con cuidado euler explícito y demás}


\textcolor{red}{¿Cómo hacer la cuenta con métodos implícitos?
Método de biyección, newton rapson, de punto fijo}





\subsubsection{Sistemas de EDOs}

y_vec' = f(y_vec, t), t>0, y_vec \in R^n
y_vec(0) = y_vec_0

\subsubsubsection{Problemas modelo en R^m}
(ec.)
Condiciones de estabilidad similares a R^1 pero sobre $\rho ()$ de una matriz.

\subsubsubsection{Problemas rígidos}
\begin{itemize}
    \item Por razones de estabilidad (debido a que unas componentes varían mucho más que las demás) se necesitaría usar un h muy pequeño.
    \item Solución: si solo interesan tiempos largos --> métodos implícitos
    \item Ej: ec. de difusión
\end{itemize}

\subsubsubsection{Problema no lineal}
y_vec' = f_vec(y_vec,t) != A y_vec
Linealización sobre el esquema numérico


\subsubsection{Normas}
Existen distintas normas, aunque están "relacionadas" por la propiedad de equivalencia entre normas.
\begin{itemize}
    \item Normas vectoriales
    \item Normas matriciales
    \item Normas matriciales inducidas
\end{itemize}


\subsection{Solución numérica de EDPs}

\subsubsection{Procedimiento estándar}
\begin{enumerate}
    \item Discretizar el dominio
    \item Semidiscretizar la ecuación diferencial, planteando un esquema para las variables espaciales
    \item Se obtiene un sistema de EDOs \textcolor{red}{Esto es gral?}
\end{enumerate}


\subsubsection{''Caballitos de batalla''}
Ecuación de advección

Ecuación de difusión

\subsubsection{Consistencia}
\begin{itemize}
    \item Si al poner la sol exacta en la ec en diferencias el residuo tiende a cero cuando $\Delta t, \Delta x \leftarrow 0$.
    \item Permite hacer elecciones adecuadas de los parámetros del método numérico para disminuir el error, aunque hay que verificar estabilidad bajo esas condiciones
    \item Teorema de Lax-Richtryer sobre consistencia en un problema lineal
\end{itemize}

\subsubsection{Estabilidad}
\begin{itemize}
    \item Existen distintas definiciones
    \item Métodos para evaluar la estabilidad en EDPs
    \begin{itemize}
        \item Método matricial: exacto pero costoso
        \item Método de Von Neumann: simple, asume varias hipótesis
        \item Método del nro de onda modificado: más gral que el anterior, asume las mismas hipótesis
    \end{itemize}
\end{itemize}

\subsection{Problemas multidimensionales}
Se pueden aplicar métodos explícitos (caros para problemas stiff) o implícitos. A mayor dimensión, mayor costo computacional

\subsubsection{Resolución eficiente de problemas $A \vec{u}^{n+1} = \vec{f}^n$}
Existen distintos métodos

\begin{itemize}
    \item Método directo
    \begin{itemize}
        \item Muy eficiente si la malla es uniforme
        \item Ej: Transformada discreta de Fourier. Limitada para condiciones de borde y grillas particulares
    \end{itemize}
    \item Métodos iterativos: 
    \begin{itemize}
        \item Más gral
        \item Ej: multigrid
    \end{itemize}
    \item Método de Factorización aproximada
    \begin{itemize}
        \item Factorizar los operadores en diferencias finitas, manteniendo el orden, para trabajar con problemas unidimensionales
        \item Útil para problemas difusivos
        \item En el camino es necesario introducir aproximaciones y, por lo tanto, hay que estudiar consistencia y estabilidad.
    \end{itemize}
\end{itemize}


\textcolor{red}{Llegué hasta esquemas mixtos sin incluir}

\bibliography{Chehade_resumen.bib}

\end{document}





